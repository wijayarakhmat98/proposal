\begin{Abstrak}

Tugas akhir ini bertujuan untuk memformalisasi isomorfisma antara aljabar matriks dan aljabar Clifford, khususnya \icm{}, menggunakan Lean \thpr{}. Aljabar Clifford menyediakan struktur yang menggeneralisasi bilangan kompleks dan kuaternion, dengan aplikasinya yang penting di berbagai bidang seperti analisis diferensial, analisis harmonik, geometri, dan fisika matematis. Konstruksi aljabar ini telah diformalisasikan dalam Lean oleh Wieser, beserta beberapa isomorfisma ke struktur matematika lainnya, namun isomorfisma spesifik yang dibahas dalam penelitian ini belum diformalisasikan. Seiring dengan semakin kompleksnya pembuktian matematika, verifikasi formal menjadi semakin penting untuk memastikan kebenaran hasil melalui validasi yang dibantu oleh komputer.

\katakunci{Aljabar Clifford, Aljabar Geometrik, Aljabar Matriks, Lean}

\end{Abstrak}

\begin{Abstract}

This thesis aims to formalize the isomorphism between matrix algebras and Clifford algebras, specifically \icm{}, within the Lean \thpr{}. Clifford algebras provide a powerful framework that generalizes complex numbers and quaternions, having significant applications in areas such as differential analysis, harmonic analysis, geometry, and mathematical physics. Their construction has already been formalized in Lean by Wieser, along with several isomorphisms to other mathematical structures, but the specific isomorphism in question has yet to be formalized. As mathematical proofs grow more complex, formal verification is becoming increasingly important for ensuring the correctness of results through computer-assisted validation.\keywords{Clifford Algebra, Geometric Algebra, Matrix Algebra, Lean}

\end{Abstract}