\makeatletter
\def\cleardoublepage{\clearpage%
	\if@twoside
	\ifodd\c@page\else
	\vspace*{\fill}
	\hfill
	\begin{center}
		\emph{ }
	\end{center}
	\vspace{\fill}
	\thispagestyle{empty}
	\newpage
	\if@twocolumn\hbox{}\newpage\fi
	\fi
	\fi
}
\makeatother
\theoremstyle{definition}
\newtheorem{defn}{Definisi}[section]
\theoremstyle{plain}
\newtheorem{teo}[defn]{Teorema}
\newtheorem{thm}{Teorema}[section]
\newtheorem{lemma}[defn]{Lemma}
\newtheorem{lemmas}[thm]{Lemma}
\newtheorem{cor}[defn]{Akibat}
\theoremstyle{definition}
\newtheorem{con}[defn]{Contoh}
\newtheorem{prop}[defn]{Proposisi}
\renewcommand{\proofname}{Bukti}
\renewcommand{\thethm}{\arabic{chapter}.\arabic{thm}}

\newcommand{\norm}[1]{\left\|#1\right\|} % Fungsi norm (||x||)

\newcommand\firstPar{0.75cm} % Indentasi 0.75cm pada tiap paragraf (manual untuk hspace)
\setlength{\parindent}{0.75cm} % Indentasi 0.75cm pada tiap paragraf

\usepackage{fancyhdr}
\pagestyle{fancy}
\renewcommand{\headrulewidth}{0pt}
\fancyhf{}
\usepackage{ifthen}
\fancyfoot[R]{\thepage}

\usepackage[labelsep=quad]{caption}
\captionsetup[table]{skip=5pt}

\usepackage{multirow}
\usepackage{longtable}

%%% Pewarnaan code
\usepackage{color}
\usepackage{listings}

\usepackage{afterpage}

\definecolor{codegreen}{rgb}{0,0.6,0}
\definecolor{codeblack}{rgb}{0,0,0}
\definecolor{codepurple}{rgb}{0.58,0,0.82}
\definecolor{backcolour}{rgb}{0.95,0.95,0.92}

\lstdefinestyle{mystyle}{
    commentstyle=\color{codegreen},
    keywordstyle=\color{magenta},
    numberstyle=\tiny\color{codeblack},
    stringstyle=\color{codepurple},
    basicstyle=\ttfamily\footnotesize,
    breakatwhitespace=false,
    breaklines=true,
    captionpos=b,
    keepspaces=true,
    numbers=left,
    numbersep=5pt,
    showspaces=false,
    showstringspaces=false,
    showtabs=false,
    tabsize=2
}
%%% Pewarnaan code

\hypersetup{ % Merubah warna link
    colorlinks,
    linkcolor={black},
    citecolor={black},
    urlcolor={black}
}

% \tolerance=1
% \emergencystretch = \maxdimen
% \hyphenpenalty=10000
% \hbadness=1000
