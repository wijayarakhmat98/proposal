\section{Penelitian Terdahulu}

Pada tahun 1960-an, Porteous menulis buku berjudul Topological Geometry, yang bertujuan untuk mempromosikan pendekatan kalkulus diferensial tanpa basis untuk fungsi-fungsi variabel banyak. Menariknya, hampir secara kebetulan, buku tersebut mencakup bagian penting tentang aljabar Clifford—sebuah generalisasi dari kuaternion yang pada waktu itu masih belum banyak dikenal. Bagi pembaca yang tertarik untuk menggali lebih dalam tentang aljabar Clifford, \citeauthor{Porteous1995} kemudian menerbitkan Clifford Algebras and the Classical Groups pada tahun \citeyear{Porteous1995}, yang memberikan penjelasan lebih mendalam mengenai topik ini.

Dalam buku Clifford Algebras and Spinors karya \cite{Lounesto2001}, dibahas struktur dan sifat-sifat aljabar Clifford, serta bagaimana aljabar ini terhubung dengan ruang spinor. Dalam buku ini, dibahas pula berbagai macam isomorfisma aljabar Clifford dengan struktur-struktur lainnya, termasuk pada matriks. Selain itu, karya \citeauthor{Lounesto2001} juga sangat relevan ketika membahas generalisasi dimensi yang lebih tinggi dari struktur aljabar, terutama dalam bidang-bidang seperti mekanika kuantum, relativitas, dan geometri diferensial. Penjelasan rinci tentang peran Clifford algebras dalam mendefinisikan representasi spinor, serta cara pengolahan masalah topologi dan geometri.

Buku Clifford Algebras: An Introduction karya \cite{Garling2011} merupakan sumber yang sangat baik untuk menggali lebih dalam mengenai aplikasi aljabar Clifford. Buku ini secara efektif menghubungkan aljabar abstrak dengan aplikasi-aplikasi dunia nyata, khususnya dalam fisika dan geometri. Garling menggambarkan bagaimana aljabar Clifford digunakan untuk mendeskripsikan spinor dalam mekanika kuantum, rotasi dalam relativitas khusus, dan objek geometri dalam geometri diferensial. Selain itu, relevansi aljabar Clifford dalam bidang-bidang seperti robotika dan komputer grafis juga dibahas, menjadikan buku ini referensi yang berharga untuk memahami baik aspek teoretis maupun praktis dari topik ini.

Theorem Proving in Lean karya \cite{Avigad2024} memberikan pengantar yang praktis untuk menggunakan Lean proof assistant dalam memformalisasi matematika. Buku ini mencakup logika dasar serta teori matematika yang lebih maju, dengan contoh dan latihan yang jelas yang menunjukkan cara menulis, memverifikasi, dan memeriksa bukti formal menggunakan Lean. Buku ini menjadi sumber yang berguna bagi mereka yang tertarik dengan verifikasi formal dan pembuktian teorema otomatis, khususnya dalam bidang matematika, ilmu komputer, dan verifikasi perangkat lunak.

Mathematics in Lean karya \cite{Avigad2025} merupakan sumber yang baik untuk eksplorasi lebih mendalam tentang penggunaan taktik dalam Lean. Buku ini berfokus pada penggunaan taktik untuk membangun dan memanipulasi ekspresi matematika kompleks, dengan pendekatan praktis dalam memformalisasi matematika menggunakan Lean. Penekanan diberikan pada cara memandu Lean dalam membangun bukti formal melalui instruksi taktik, menjadikannya referensi yang ideal untuk berinteraksi dengan Lean secara dinamis.

Esai \cite{Palmer2020} membahas ide dasar dari set theory dan type theory, dengan fokus pada dependent type theory sebagai alternatif dari set theory klasik dalam memformalkan matematika. Dijelaskan bahwa Lean menggunakan dependent type theory, yang menggabungkan logika intuisionistik melalui korespondensi "proposisi-sebagai-tipe", memberikan keuntungan praktis, terutama dalam pemrograman, karena menggunakan spesifikasi tipe yang lebih tepat. Meskipun Lean didasarkan pada logika intuisionistik, penalaran klasik masih bisa diterapkan jika diperlukan, sehingga memberikan fleksibilitas dalam formalisasi matematika. Esai ini menunjukkan bagaimana peralihan dari set theory ke type theory dapat mengubah cara matematika diformalkan dengan pendekatan yang berbeda.

Disertasi \cite{Wieser2024}, Formalizing Clifford Algebras and Related Constructions in the Lean Theorem Prover, adalah formalisasi struktur aljabar Clifford dalam Lean. Ini menjadi landasan yang penting pada tugas akhir ini, sebab untuk melakukan formalisasi \icm{}, diperlukan adanya implementasi struktur aljabar Clifford yang dapat digunakan. Pada disertasi tersebut, telah dilakukan pula beberapa formalisasi isomorfisma aljabar Clifford pada struktur lainnya. Strategi formalisasi yang digunakan oleh Wieser dapat dijadikan inspirasi dalam upaya formalisasi di tugas akhir ini.
